\section{Document Scope}
\label{document-scope}

This document is not intended to be a comprehensive description of the
\ScopeSim{} environment.  Rather it aims to introduce the reader to
the elements that make up \ScopeSim{} and directs the reader towards
the online documentation for each of the packages, should the reader
wish to dive deeper into the material (see
Table~\ref{tbl-list-of-packages}).


\section{Introduction}
\label{introduction}

\ScopeSim{} is a modular and flexible suite of Python packages that
enable many common astronomical optical systems (observatory
site/telescope/instrument) to be simulated.  The suite of packages can
be used by a wide audience for a variety of purposes; from the
astronomer interested in simulating reduced observational data, to a
developer needing raw calibration data for testing the data reduction
software.

\ScopeSim{} achieves this level of flexibility by defining strict
interfaces between the packages. The \ScopeSim{} engine package is
completely instrument and object agnostic.  All information and data
relating to any specific optical configuration is kept exclusively in
the instrument packages hosted in the instrument reference database
(\IRDB{}).  The description of the on-sky source is kept exclusively
within the target templates package (\ScopeSimtemplates{}).  Finally,
the engine makes no assumptions about what it is observing until
run-time.

%% OC: This paragraph seems out of place in the MICADO FDR documentation
% But why does the community need yet another instrument simulator?
% Until now, most instrument consortia have developed, or are continuing
% to develop their own simulators\cite{hsim, schmalzl2012, simcado2016,
%   simcado2019}.  The general consensus is that every new instrument is
% sufficiently different from anything that has been previously
% developed, that it would make little sense to adapt already existing
% code.  This statement is true to some extent.  Every new instrument
% must differ in some way from all existing instruments in order for it
% to be useful to the astronomical community.  However when looked at
% from a global perspective, every optical system is comprised primarily
% of elements common to many other systems.  Atmospheric emission,
% mirror reflectivities, filter transmission curves, point spread
% functions, read-out noise, detector linearity, hot pixels, are just a
% few of the effects and artefacts that every astronomical optical
% system contains.  Furthermore, while the amplitude and shape of each
% effect differs between optical systems, there are still commonalities
% in the way each effect can be described programmatically.

\ScopeSim{}'s main goal is to provide a framework for modeling
(almost) any astronomical optical system by taking advantage of the
commonalities between instruments.
% What
% \lstinline{Astropy}\cite{astropy1, astropy2} has done for the general
% Python landscape in astronomy, \ScopeSim{} aims to do for the
% instrument simulator landscape.



%%% Local Variables:
%%% mode: latex
%%% TeX-master: "../main"
%%% End:
