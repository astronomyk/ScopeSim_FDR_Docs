\section{Conclusion}
\label{conclusion}

\ScopeSim{} is a flexible multipurpose instrument data simulation framework built in Python.
It enables both raw and ideal observation data to be simulated for a wide range of telescopes and instruments quickly and efficiently on standard personal computers.
This is achieved by keeping the instrument model data, descriptions of the target objects, and simulation engine strictly separated.
The three main packages are the \ScopeSim{} engine, a library of target templates, and the instrument reference database.
Several existing and future telescope and instrument systems have been already been implemented, with more to come in the future.
For example, work is steadily progressing on the instrument packages for the MICADO and METIS\cite{metis2018} instruments at the ELT.
%All the code and data is open source and hosted on Github.

\textbf{Community involvement is highly encouraged!}
The whole ScopeSim framework is open source and the developers welcome any contributions, both code and comments, by members of the astronomical community.
The astronomical object templates package is one area which will benefit greatly from community contributions.
There is a wide variety of astronomical objects for which the authors have not yet created templates.
Galaxy clusters, gravitational lenses, supernovae, exoplanets, solar system objects are all still missing from the \ScopeSimtemplates{} package.
The instrument reference database also currently only contains the instruments directly relevant to the authors, i.e. MICADO, METIS, HAWKI, and the LFOA.
There is no limit to the size of telescopes or number of instruments that can be hosted on the server.
Readers interested in submitting a package for their own telescope or instrument are very welcome to make a pull request on the \IRDB{} Github page.
