\section{Executive Summary}
\label{sec:summary}

\ScopeSim{} is a flexible multipurpose instrument data simulation
framework built in Python.  It enables both raw and reduced
observation data to be simulated for a wide range of telescopes and
instruments quickly and efficiently on a personal computer.  The
software is currently being used to generate simulated raw input data
for developing the data reduction pipelines for the MICADO and METIS
instruments at the ELT.  The \ScopeSim{} environment consists of three
main packages which are responsible for providing on-sky target
templates (\ScopeSimtemplates{}), the data to build the optical models
of various telescopes and instruments (instrument reference database),
and the simulation engine (\ScopeSim{}).  This strict division of
responsibilities allows \ScopeSim{} to be used to simulate observation
data for many different instrument and telescope configurations for
both imaging and spectroscopic instruments.  \ScopeSim{} has been
built to avoid redundant calculations wherever possible.  As such it
is able to deliver simulated observations on time scales of seconds to
minutes.  All the code and data is open source and hosted on Github.
%The community is also most welcome, and indeed encouraged to
%contribute to code ideas, target templates, and instrument packages.

%%% Local Variables:
%%% mode: latex
%%% TeX-master: "../main"
%%% End:
