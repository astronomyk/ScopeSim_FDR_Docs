

\section{SpeXtra%
  \label{spextra}%
}

\texttt{speXtra} is a tool to manage and manipulate astronomical spectra. 

Documentation: \url{https://spextra.readthedocs.io/}

Code Base: \url{https://github.com/miguelverdugo/speXtra}

Continuous integration: \url{https://travis-ci.org/github/miguelverdugo/speXtra}

Author: Miguel Verdugo


\subsection{Functionality%
  \label{functionality}%
}

\texttt{speXtra} packages several \texttt{synphot} workflows
workflows in simple-to-use methods that allow the user 
to manipulate astronomical spectra. For example, the user
can extract the magnitude from an astronomical spectrum or 
scale the specrum, the spectrum can be redshifted or 
blueshifted or smoothed with a velocity kernel, add emission or
absorption lines, rebin the spectra or correct it for an
extinction curve. Etc.

\texttt{speXtra} does not perform measurements, with the sole 
exception of extracting magnitudes within a band. 

\texttt{speXtra} also comes with a built-in database of spectral templates 
that should cover most of the science cases. Loading these templates is as
easy as typing \texttt{Spextrum("library_name/template_name")}



\subsection{Installation}

To install \texttt{speXtra} simply type

\texttt{pip install spextra}

 


\subsection{Examples%
  \label{examples}%
}
