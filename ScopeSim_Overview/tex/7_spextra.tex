

\section{SpeXtra%
  \label{spextra}%
}

\texttt{speXtra} is a tool to manage and manipulate astronomical spectra.

Documentation: \url{https://spextra.readthedocs.io/}

Code Base: \url{https://github.com/miguelverdugo/speXtra}

Continuous integration: \url{https://travis-ci.org/github/miguelverdugo/speXtra}

Author: Miguel Verdugo


\subsection{Functionality%
  \label{functionality}%
}

\texttt{speXtra} packages several \texttt{synphot} workflows in simple-to-use methods that allow the user
to manipulate astronomical spectra. For example, the user
can extract the magnitude from an astronomical spectrum or
scale the spetcrum, the spectrum can be redshifted or
blueshifted or smoothed with a velocity kernel, add emission or absorption lines, rebin the spectra or correct it for an extinction curve. Etc.

\texttt{speXtra} does not perform measurements, with the sole exception of extracting magnitudes within a passband.

\texttt{speXtra} also comes with a built-in database of spectral templates  that cover many possible user cases. Loading these templates is as easy as typing \texttt{Spextrum('library\textbackslash{}\_name/template\textbackslash{}\_name')}


\subsubsection{Database%
  \label{database}%
}

The \texttt{speXtra} database contains libraries of spectral templates, extinction curves and filter systems.
The data is downloaded on-the-fly when requested and kept cached in the local hard drive for future use.

The scheme is flexible and additional data can be added.
In the following a short summary of the database contains is provided.


\subsubsection{Spectral Templates%
  \label{spectral-templates}%
}

At the time of writing the following libraries are included in \texttt{speXtra}.
Other can be added at request of the user.

\begin{itemize}
\item The Kinney-Calzetti Spectral Atlas of Galaxies

\item Pickles Stellar Library

\item SDSS galaxy composite spectra

\item IRTF spectral library

\item AGN templates

\item Emission line nebulae

\item Galaxy SEDs from the UV to the Mid-IR

\item Kurucz 1993 Models (subset)

\item Supernova Legacy Survey templates (subset)

\item Flux/Telluric standards with X-Shooter

\item High-Resolution Spectra of Habitable Zone Planets (example)
\end{itemize}


\subsubsection{Extinction Curves%
  \label{extinction-curves}%
}

Extinction curves provided with the database.

\begin{itemize}
\item Gordon LMC/SMC extinction curves

\item Cardelli MW extinction curves

\item Calzetti starburst attenuation curve
\end{itemize}


\subsubsection{Filter Systems%
  \label{filter-systems}%
}

\begin{itemize}
\item MICADO filter system
\end{itemize}


\subsubsection{Installation%
  \label{installation}%
}

To install \texttt{speXtra} simply type:

\begin{quote}
\begin{alltt}
pip install spextra
\end{alltt}
\end{quote}


\subsection{Examples%
  \label{examples}%
}
