

\section{The ScopeSim engine%
  \label{the-scopesim-engine}%
}

Documentation: \url{https://scopesim.readthedocs.io/}

Code Base: \url{https://github.com/astronomyk/ScopeSim}

Continuous integration: \url{https://travis-ci.org/github/astronomyk/ScopeSim}

Author: Kieran Leschinski


\subsection{How the ScopeSim engine works%
  \label{how-the-scopesim-engine-works}%
}

The scopesim engine is the core of the scopesim environment.
At the heart of scopesim are two major concepts.

\begin{itemize}
\item The observed output is the target input plus a series of optical artefacts

\item The effect of each optical artefact is independent of all other artefacts
\end{itemize}

If we follow these to concepts to their logical conclusion we end up with a situation where optical artefacts can be treated as a series of \textquotedbl{}lego bricks\textquotedbl{} and any digital optical model can be constructed much in the same way as a lego model; by stacking the correct combination of effect \textquotedbl{}bricks\textquotedbl{} together in the right order.

The ScopeSim engine is therefore comprised of two types of objects: a collection of \texttt{Effect} object subclasses, and a series of \textquotedbl{}management\textquotedbl{} classes.

The \texttt{Effect} object is a lightweight class that acts as a simple operator class.
Object goes in, object comes out.
Albeit with the flux distribution slightly altered in one way or another.
Here the object in question can be any one of the 4 main flux distribution classes:

\begin{itemize}
\item \texttt{Source}: (x, y, lambda)

\item \texttt{FieldOfView}: (x, y, lambda\_0)

\item \texttt{ImagePlane}: (x, y, sum(lambda))

\item \texttt{DetectorPlane}: (x, y, e-)
\end{itemize}

\DUadmonition[warning]{
\DUtitle[warning]{Warning}

finish section
}


\subsection{Optical system capabilities%
  \label{optical-system-capabilities}%
}

\begin{itemize}
\item Imaging

\item Spectroscopy (LS, MOS, IFU)

\item Simultaneous readouts on multiple image planes
\end{itemize}


\subsection{Explanation of Effect objects%
  \label{explanation-of-effect-objects}%
}

\begin{itemize}
\item effects are operators, what is passed, is returned

\item 
\begin{description}
\item[{effect classes can alter the object in a variety of ways:}] \leavevmode 
\begin{itemize}
\item intensity (mult, add, sub) in x,y and/or lambda

\item shifts

\item convolutions

\item spreads
\end{itemize}

\end{description}
\end{itemize}

effects are applied one after the other, first lam, then xylam, then xy
- effects can also be electronic in nature,
- examples of effects, psf, linearity, exposure


\subsection{Explanation of observation workflow%
  \label{explanation-of-observation-workflow}%
}

\begin{itemize}
\item pseudo code for loop

\item description of 4 objects, and why they are important
\end{itemize}


\subsection{Documentation%
  \label{documentation}%
}

\begin{itemize}
\item Tutorials on read the docs
\end{itemize}
