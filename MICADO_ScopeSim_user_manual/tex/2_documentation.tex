

\section{Essentials: Documentation and Downloads%
  \label{essentials-documentation-and-downloads}%
}


\subsection{Online Documentation%
  \label{online-documentation}%
}

Online documentation for the main packages in the ScopeSim environment can be found here:

\begin{itemize}
\item ScopeSim: \url{https://scopesim.readthedocs.io/en/latest/}

\item ScopeSim\_Templates: \url{https://scopesim-templates.readthedocs.io/en/latest/}

\item IRDB: \url{https://github.com/astronomyk/irdb}
\end{itemize}

The original SimCADO package is described here:

\begin{itemize}
\item SimCADO: \url{https://simcado.readthedocs.io/en/latest/}
\end{itemize}

\DUadmonition[note]{
\DUtitle[note]{Note}

In the near future we will release a wrapper for the ScopeSim engine and the MICADO instrument package.

The doumentation for this will be added to the original \href{https://simcado.readthedocs.io/en/latest/}{SimCADO} read-the-docs page
}


\subsection{Downloading ScopeSim and the MICADO package%
  \label{downloading-scopesim-and-the-micado-package}%
}

The ScopeSim engine is installed using pip:

\begin{quote}
\begin{alltt}
\begin{lstlisting}[frame=single]
$ pip install scopesim
\end{lstlisting}
\end{alltt}
\end{quote}

The casual user will also probably want to install the templates package, which contains helper functions for generating descriptions of on-sky targets like elliptical galaxies or star clusters:

\begin{quote}
\begin{alltt}
\begin{lstlisting}[frame=single]
$ pip install scopesim_templates
\end{lstlisting}
\end{alltt}
\end{quote}

Once ScopeSim is available to the local Python (version >= 3.5) installation, the user must download \textbf{ALL} the required instrument packages from the server:

\begin{quote}
\begin{alltt}
\begin{lstlisting}[frame=single]
from scopesim.server import download_package
scopesim.download_package(["locations/Armazones",
                           "telescopes/ELT",
                           "instruments/MAORY",
                           "instruments/MICADO"])
\end{lstlisting}
\end{alltt}
\end{quote}

\DUadmonition[note]{
\DUtitle[note]{Note}

There are two (2) MICADO packages available: \texttt{MICADO} and \texttt{MICADO\_Sci}.
}

For those interested in quick results, \texttt{MICADO\_Sci} provides a reduced version of the MICADO package that contains all the major effects expected from the MICADO optical system.
For those more concerned with accuracy, the standard \texttt{MICADO} package contains all expected optical effects.
\texttt{MICADO} was originally developed for the development of the reduction pipeline, and therefore contains many effects that are beyond the scope of normal science case feasability studies.


\subsection{Primary vs Support packages%
  \label{primary-vs-support-packages}%
}

\texttt{MICADO} and \texttt{MICADO\_Sci} are primary packages.
This means they contain detector modules that enable an on-sky target to be observed

Armazones, ELT, and MAORY are support packages.
They do not contain detector modules.

Just like in real life, observing with only MICADO would be a difficult task.
Therefore we encourage the user to also download the support packages needed by MICADO.
