

\section{Controlling a simulation%
  \label{controlling-a-simulation}%
}


\subsection{A note on the two MICADO packages%
  \label{a-note-on-the-two-micado-packages}%
}

It should be noted that there are two MICADO packages available:

\begin{enumerate}
\item MICADO\_Sci: optimised for quick run times to enable rapid iterations of science feasibility use cases

\item MICADO: the generalised package containing all the known optical effects needed by the pipeline development team.
\end{enumerate}

In general science team members are recommended to use the MICADO\_Sci package as the generalised MICADO is slower and exposes many more configuration parameters to the user.

Both packages \textbf{still} require all upstream support packages, namely: Armazones, ELT, and MAORY.


\subsection{Available MICADO observing modes%
  \label{available-micado-observing-modes}%
}

As mentioned in the previous sections there are two MICADO models available: MICADO\_Sci and MICADO.
The desired model is selected with the \texttt{use\_instrument=} parameter:

\begin{quote}
\begin{alltt}
\begin{lstlisting}[frame=single]
cmd = scopesim.UserCommands(use_instrument="MICADO_Sci", ...)
\end{lstlisting}
\end{alltt}
\end{quote}

The choice of observing modes can be set when the \texttt{UserCommand} object is initialised by passing a list of mode names to the \texttt{set\_mode=} parameter.
Note that multiple keywords can be passed as different modes for different optical sections can be combined.
For example, using the MAORY relay optics with the MICADO wide-field imaging mode requires the following input:

\begin{quote}
\begin{alltt}
\begin{lstlisting}[frame=single]
cmd = scopesim.UserCommands(..., set_modes=["MCAO", "IMG_4mas"])
\end{lstlisting}
\end{alltt}
\end{quote}

In contrast, to use the MICADO internal SCAO mode with the zoom imaging mode, the following input is required:

\begin{quote}
\begin{alltt}
\begin{lstlisting}[frame=single]
cmd = scopesim.UserCommands(..., set_modes=["SCAO", "IMG_1.5mas"])
\end{lstlisting}
\end{alltt}
\end{quote}

The following table lists all available mode keywords for MICADO.

\setlength{\DUtablewidth}{\linewidth}
\begin{longtable*}[c]{|p{0.15\DUtablewidth}|p{0.817\DUtablewidth}|}
\hline
\textbf{%
Keyword
} & \textbf{%
Description
} \\
\hline
\endfirsthead
\hline
\textbf{%
Keyword
} & \textbf{%
Description
} \\
\hline
\endhead
\multicolumn{2}{c}{\hfill ... continued on next page} \\
\endfoot
\endlastfoot

MCAO
 & 
MAORY relay optics model is included
 \\
\hline

SCAO
 & 
MICADO SCAO relay optics module is included
 \\
\hline

IMG\_4mas
 & 
Wide-field imaging optics included. Images have 4mas pixel scale.
 \\
\hline

IMG\_1.5mas
 & 
Zoom imaging optics included. Images have 1.5mas pixel scale.
 \\
\hline

SPEC
 & 
'MICADO\_Sci' package only. Produces rectified spectroscopic images.
 \\
\hline

SPEC\_3000x20
 & 
'MICADO' package only. Produces full raw detector traces for the short-narrow slit (3000x20mas)
 \\
\hline

SPEC\_3000x50
 & 
'MICADO' package only. Produces full raw detector traces for the short-wide slit (3000x50mas)
 \\
\hline

SPEC\_15000x50
 & 
'MICADO' package only. Produces full raw detector traces for the long-wide slit (15000x50mas)
 \\
\hline
\end{longtable*}

\DUadmonition[note]{
\DUtitle[note]{Note}

While the imaging mode names are consistent between the MICADO\_Sci and MICADO packages, the spectroscopy mode names differ.
}


\subsection{Setting top level observation parameters%
  \label{setting-top-level-observation-parameters}%
}

The \texttt{UserCommands} object is built on several hierarchical (YAML-style) nested dictionaries which contain all the information needed to build the optical model of MICADO.
Checking and changing parameters in these dictionaries is done using the so-called \textquotedbl{}bang-string\textquotedbl{} notation.
Bang-strings shorten the syntax for accessing nested dictionaries.
E.g. \texttt{cmd{[}\textquotedbl{}!OBS.ndit\textquotedbl{}{]}} is the equivalent of \texttt{cmd{[}\textquotedbl{}OBS\textquotedbl{}{]}{[}\textquotedbl{}ndit\textquotedbl{}{]}}

Alternatively, a dictionary of bang-string keyword-value pairs can be passed to the \texttt{UserCommands} object when initialising via the \texttt{properties} keyword:

\begin{quote}
\begin{alltt}
\begin{lstlisting}[frame=single]
scopesim.UserCommands(properties={"!OBS.dit": 60, "!OBS.ndit": 10})
\end{lstlisting}
\end{alltt}
\end{quote}

The following table describes the top level parameter categories.

\setlength{\DUtablewidth}{\linewidth}
\begin{longtable*}[c]{|p{0.060\DUtablewidth}|p{0.880\DUtablewidth}|}
\hline
\textbf{%
Alias
} & \textbf{%
Description
} \\
\hline
\endfirsthead
\hline
\textbf{%
Alias
} & \textbf{%
Description
} \\
\hline
\endhead
\multicolumn{2}{c}{\hfill ... continued on next page} \\
\endfoot
\endlastfoot

OBS
 & 
Instrument specific parameters. E.g. exposure time and number (DIT, NDIT), filter name, etc
 \\
\hline

ATMO
 & 
Atmospheric specific parameters. E.g. air temperature, background brightness, location, etc
 \\
\hline

TEL
 & 
Telescope specific parameters. E.g. Dome temperature, etc
 \\
\hline

RO
 & 
Relay optics specific parameters. E.g. RO module temperature, etc
 \\
\hline

INST
 & 
Instrument specific parameters. E.g. plate scale, strehl ratio (MICADO\_Sci only), etc
 \\
\hline

DET
 & 
Detector specific parameters. E.g. windowing dimensions and position (MICADO\_Sci only), etc
 \\
\hline

SIM
 & 
Instrument specific parameters. E.g. spectral range, computing parameters, file locations, etc
 \\
\hline
\end{longtable*}

\DUadmonition[note]{
\DUtitle[note]{Note}

Important parameters like which filter to use and how long to observe for are generally kept in the \texttt{\textquotedbl{}!OBS\textquotedbl{}} dictionary
}

The whole parameter dictionary can be shown by simply printing the \texttt{UserCommand} object:

\begin{quote}
\begin{alltt}
\begin{lstlisting}[frame=single]
print(cmd)
\end{lstlisting}
\end{alltt}
\end{quote}

Specific categories can be shown by using the category alias in the bang string format:

\begin{quote}
\begin{alltt}
\begin{lstlisting}[frame=single]
print(cmd["!OBS"])
\end{lstlisting}
\end{alltt}
\end{quote}

A specific parameter (or sub-group thereof) can be shown by following the hierarchy in bang-string format:

\begin{quote}
\begin{alltt}
\begin{lstlisting}[frame=single]
print(cmd["!INST.psf.strehl"])
\end{lstlisting}
\end{alltt}
\end{quote}

Parameter values can also be set in this manner.
For example, to set which filter to use with MICADO, we call the \texttt{\textquotedbl{}OBS.filter\_name} keyword:

\begin{quote}
\begin{alltt}
\begin{lstlisting}[frame=single]
cmd["!OBS.filter_name"] = "Ks"
\end{lstlisting}
\end{alltt}
\end{quote}

\DUadmonition[note]{
\DUtitle[note]{Note}

The full list of filters available in MICADO is kept in the \texttt{OpticalTrain} object. This will be discussed below.
}

The following list contains some commonly used parameters for the \texttt{MICADO\_Sci} package:

\begin{quote}
\begin{alltt}
\begin{lstlisting}[frame=single]
MICADO_Sci
==========

OBS.filter_name: "J"                    # MICADO filter name
OBS.dit: 60                             # [s] Length of exposure
OBS.ndit: 1                             # Number of exposures to stack
OBS.airmass: 1.2

ATMO.background.filter_name: "Ks"
ATMO.background.value: 13.6             # Amplitude of background flux
ATMO.background.unit: "mag"             # Unit of background flux

INST.psf.strehl: 0.8                    # Desired Strehl ratio (as far as
                                        #   physically possible)
INST.psf.wavelength: "Ks"               # Either wavelength [um] or common
                                        #   broadband filter name
INST.aperture.x: 0                      # [arcsec] (MICADO_Sci only) Slit
                                        #   position relative to FOV centre
INST.aperture.y: 0
INST.aperture.width: 3                  # [arcsec] (MICADO_Sci only)
                                        #    The slit dimensions
INST.aperture.height: 0.05

DET.width: 1024                         # [pixel] Detector dimensions
DET.height: 1024
DET.x: 0                                # [pixel] Position of window centre
                                        #   relative to detector plane centre
DET.y: 0

SIM.spectral.wave_min: 0.7              # [um] Spectral borders of simulation
SIM.spectral.wave_mid: 1.6              # [um] Central wavelength for SPEC mode
SIM.spectral.wave_max: 2.5
SIM.sub_pixel.flag: False               # Turn on for astrometry observation
SIM.random.seed: 9001                   # Random seed for numpy
SIM.file.use_cached_downloads: True     # Turn off when checking for updates
\end{lstlisting}
\end{alltt}
\end{quote}

The spectroscopy mode of the \texttt{MICADO\_Sci} package contains a configurable slit and detector window.
This allows the user to choose only the spectral range that is relevant to their science case.
Restricting the spectral range like this speeds up the computation time by order and allows for fast iterations of science use cases.
The pipeline-oriented \texttt{MICADO} package contains fixed slit dimensions and spectral trace layouts.
Simulations run using this setup are slow and memory intensive.


\subsection{The MICADO filter settings%
  \label{the-micado-filter-settings}%
}

To set the filter name, we use the \texttt{!OBS.filter\_name} bang-string.

To find out which filters are included in the MICADO\_Sci and MICADO packages, we must create an OpticalTrain model for MICADO:

\begin{quote}
\begin{alltt}
\begin{lstlisting}[frame=single]
cmd = scopesim.UserCommands(use_instrument="MICADO_Sci")
micado = scopesim.OpticalTrain(cmd)
\end{lstlisting}
\end{alltt}
\end{quote}

In the \texttt{MICADO\_Sci} package all filters are contained in a single \texttt{filter\_wheel} object:

\begin{quote}
\begin{alltt}
\begin{lstlisting}[frame=single]
micado["filter_wheel"].data["name"]
\end{lstlisting}
\end{alltt}
\end{quote}

For the pipeline-oriented \texttt{MICADO} package, the filters are distributed among the three filter wheels:
\texttt{filter\_wheel\_1}, \texttt{filter\_wheel\_2}, and \texttt{pupil\_wheel}.
Hence there is no central list of all filters for the \texttt{MICADO} package.


\subsection{The MICADO\_Sci PSF model%
  \label{the-micado-sci-psf-model}%
}

The \texttt{MICADO\_Sci} package uses the AnisoCADO python package to generate ELT-like PSFs on the fly.
The Strehl ratio can therefore be set (within reason) to whatever the user desires.
The global bang-string keywords for this are \texttt{!OBS.psf.strehl} and \texttt{!OBS.psf.wavelength}:

\begin{quote}
\begin{alltt}
\begin{lstlisting}[frame=single]
cmd["!OBS.psf.strehl"] = 0.35
cmd["!OBS.psf.wavelength"] = 2.15
\end{lstlisting}
\end{alltt}
\end{quote}

These settings will create an on-axis PSF with a 35\% Strehl ratio at 2.15um.
At shorter wavelengths the Strehl ratio will scale as expected.
If a Strehl ratio is physically impossible (e.g. 80\% at 0.9um), the largest PSF will be generated with the best physically possible Strehl ratio.

It is possible to also create PSF models for off-axis positions in SCAO mode.
The offset can only be specified after the \texttt{OpticalTrain} model has been created by accessing the PSF object inside the model.
The relevant MICADO \texttt{OpticalTrain} objects are \texttt{scao\_const\_psf} for SCAO observations and \texttt{maory\_const\_psf} for MCAO observations:

\begin{quote}
\begin{alltt}
\begin{lstlisting}[frame=single]
cmd = scopesim.UserCommands(use_instrument="MICADO_Sci",
                            set_modes=["SCAO", "IMG_4mas"])
cmd["!OBS.psf.strehl"] = 0.35
cmd["!OBS.psf.wavelength"] = 2.15

micado = scopesim.OpticalTrain(cmd)
micado["scao_const_psf"].meta["offset"] = [10, 0]
\end{lstlisting}
\end{alltt}
\end{quote}

This will create an on-axis PSF with a 35\% Strehl ratio at 2.15um, and then shift the 10 arcseconds off axis.
The form and Strehl ratio of the PSF will be adjusted to match what is expected for the off-axis position.

\DUadmonition[note]{
\DUtitle[note]{Note}

While the \texttt{MICADO\_Sci} PSF model is configurable, the \texttt{MICADO} PSF model is not.

The \texttt{MICADO}-package uses pre-computed PSF files. These are available on the ScopeSim server: \url{https://www.univie.ac.at/simcado/InstPkgSvr/psfs/}
}

% - Other major configuration parameters
% - filter
% - dit / ndit
% - slit size
% - zenith distance
% - psf model
