% listings style
\lstset{basicstyle=\ttfamily,
        columns=flexible,
        frame=single,
        backgroundcolor=\color{listingbg},
        captionpos=b,
        showspaces=false}

\dmdmaketitle

%\emptypage{This page was intentionally left blank}

\begin{center}
  \textbf{Change record}

  \tablehead{\hline
    \multicolumn{1}{|c|}{Issue/Rev.} &
    \multicolumn{1}{|c|}{Date} &
    \multicolumn{1}{|c|}{Section/Parag.\ affected} &
    \multicolumn{1}{|c|}{Reason/Initiation/Documents/Remarks} \\
    \hline}
  \tabletail{\hline}

  \begin{supertabular}{|l|l|l|l|}
   0.0.1 & 2020-10-15 & All & Layout initialised \\
   \hline
  \end{supertabular}

\end{center}

%\emptypage{This page was unintentionally left blank}

\setcounter{tocdepth}{3}
\tableofcontents
\cleardoublepage



\section*{Abbreviations}
\label{sec:abbreviations}

%\tablehead{%
%  \hline
%  \multicolumn{1}{|c|}{Abbreviation} &
%  \multicolumn{1}{|c|}{Meaning} \\
%  \hline}
%\tabletail{\hline}

\tablehead{}
\tabletail{}
\begin{supertabular}{ll}
  DFS & Data Flow System \\
  ELT & Extremely Large Telescope \\
  ESO & European Southern Observatory \\
  HCI & High Contrast Imaging \\
  IMG & Imaging observing modi \\
  MCAO & Multi-Conjugate Adaptive Optics \\
  PSF & Point Spread Function \\
  SCAO & Single Conjugate Adaptive Optics \\
  SPEC & Spectroscopy observing modi \\
  TBC & To Be Confirmed \\
  TBD & To Be Decided \\
\end{supertabular}


\section{Scope}
\label{sec:introduction}

This document aims to introduce the reader to the ScopeSim instrument simulator and the MICADO instrument packages that have been developed as part of the MICADO instrument simulator work package.
ScopeSim is a second generation instrument simulation environment, following on from the original SimCADO package.
In this context, the title of the document "SimCADO v2 user manual" refers to the fact that the MICADO instrument simulator is now a combination of the ScopeSim software and the associated MICADO packages.
The ScopeSim environment consists of a series of Python packages, with the ScopeSim instrument simulation engine at its core.
All packages are available within the standard python ecosystem.
The ScopeSim ecosystem maintains a very strict split between code and data.
The simulation engine and all associated software packages are completely instrument agnostic.
Instrument models can only be initialised by passing an instrument specific data package to the ScopeSim engine.

For MICADO two instrument data package were produced as input for the ScopeSim engine: ``MICADO'' and ``MICADO\_Sci''.
The original use case for the simulator was as a means of producing raw detector frames to aid in the development of the MICADO data reduction pipeline.
Hence the ``MICADO`` instrument data package contains all the elements needed to simulate realistic raw detector readout images.
The primary audience for the ``MICADO`` instrument data package is the MICADO dataflow system work package.
Many of the effect included in the ``MICADO`` package will be removed by the data reduction pipeline.
Simulating and subsequently removing these effects results in a substantial amount of redundant computational overhead for the second group of use cases for the software: science case feasibility studies.
In order to enable rapid iteration of simulations for these feasibility studies, a second instrument data package was compiled: ``MICADO\_Sci``.
``MICADO\_Sci`` contains only the optical effects that will remain in the images after the data reduction pipeline has successfully processed the images.

This document should be seen as the initial introduction to ScopeSim and the MICADO data packages.
It contains the following information:
\begin{itemize}
\item Introduction to ScopeSim environment
\item Where to find the online documentation
\item How to begin simulating MICADO observations with ScopeSim
\item How to control simulations
\item Building on-sky target object as input for ScopeSim
\item Five use case examples for the imaging and spectroscopy modes of MICADO
\end{itemize}

It should be noted that what is contained in this document is a subset of what is available in the online documentation for ScopeSim.
Yet more examples and information are available at \url{https://scopesim.readthedocs.io/en/latest/}.


\section{Applicable and Reference Documents}
\label{sec:documents}

\subsection{Applicable Documents}

\begin{center}
  \tablehead{\hline
    \multicolumn{1}{|c|}{Nr} &
    \multicolumn{1}{|c|}{Doc.\ Nr} &
    \multicolumn{1}{|c|}{Doc.\ Title} &
    \multicolumn{1}{|c|}{Issue} &
    \multicolumn{1}{|c|}{Date} \\
    \hline}
  \tabletail{\hline}

  \begin{supertabular}{|l|l|l|c|l|}
    AD\,1 & ELT-ICD-MCD-56306-0058 & SimCADO User Manual & 1.0 & 2021-04-12 \\
    AD\,2 & ELT-ICD-MCD-56306-0059 & Instrument data packages for the & & \\
    & & MICADO simulation environment & 1.0 & 2021-04-12 \\
    AD\,3 & ELT-ICD-MCD-56306-0060 & ScopeSim - A modular astronomical & & \\
    & & instrument data simulation environment & 1.0 & 2021-04-12 \\
    AD\,4 & ELT-TRE-MCD-56300-0014 & MICADO Masks, Stops, and Filters Description & 2.9 & 2020-12-04 \\
  \end{supertabular}
\end{center}

\subsection{Reference Documents}

\begin{center}
  \tablehead{\hline
    \multicolumn{1}{|c|}{Nr} &
    \multicolumn{1}{|c|}{Doc.\ Nr} &
    \multicolumn{1}{|c|}{Doc.\ Title} &
    \multicolumn{1}{|c|}{Issue} &
    \multicolumn{1}{|c|}{Date} \\
    \hline}
  \tabletail{\hline}

  \begin{supertabular}{|l|l|l|c|l|}
    \hline
    RD\,1 & ESO-257871 & MICADO Statement of Work & 2 & 2020-12-10  \\
    RD\,2 & ELT-PLA-MCD-56301-0004 & Operational Concept Description &
    2 & 2019-11-09 \\
    RD\,3 & ELT-TRE-MCD-56300-0011 & MICADO System Design and Analysis & 1.5 &
    2021-04-12 \\
    RD\,4 & ELT-ICD-MCD-56306-0050 & SimCADO: the Data Simulator for MICADO & 1.0 & 2018-09-27 \\

  \end{supertabular}
\end{center}